
The thermoelectric model constructs a profile of the sensor module operation conditions over the
lifetime of the detector in the following manner. First, assuming a reasonable set of initial
component temperatures, the module total power (including all components, but excluding the sensor
leakage power) and the sensor temperature without leakage current ($T_0$) is calculated assuming these
initial temperatures.
The initial value for the module power is used to solve for the sensor power and temperature accounting
for leakage current, using the thermal balance equation and the relationship from
Eq.~\ref{eq:leakage_current_temp_dependence}.
Using the calculated sensor leakage current and temperature, the power and temperature of the module
components are updated given the inital (year-0, month-0) startup parameters.

Next, the module conditions of the following month (year-0, month 1) are calculated. Using the component
temperatures calculated from the previous month and the operational parameters (ionizing dose and dose
rates) from month 1, the module total power (excluding sensor leakage) is again recalculated, and
subsequently the sensor temperature and leakage current are recomputed. Following this,
the module component temperatures and power values are updated. This process is repeated in one-month
steps until the final year of operation, or until a real solution for the sensor temperature does not
exist, indicating that thermal runaway conditions have been reached.

In the barrel subsystem, the above procedure is performed four separate times to
represent the conditions of the four barrel layers located at different radii from the beam axis
\footnote{
The correct module type, short-strip in the inner two layers and long-strip for the outer two
layers, is used for each layer.
}.
%
The procedure is also performed once for a module adjacent to an EOS and a representative module
of the remaining 13 without an adjacent EOS. In each case, the worst-case total ionizing dose and dose
rate is assumed for each layer, though these values are relatively consistent for a given layer.
In total, 8 modules are simulated, and they are combined in their proper proportion to simulate the
entire barrel system.

In the endcap subsystem, the total ionizing dose and dose rates vary signficantly depending on the
position of the module; furthermore, the design of each module on a petal differs significantly.
Therefore, all 36 module types (6 rings $\times$ 6 disks) are simulated independently, and combined to
represent the full endcap.


