\textit{(This section should be discussed and agreed between Graham and Yu-Heng)}

The cooling path between the sources dissipating electrical power and the cooling fluid is three-dimensional and includes components with orthotropic thermal conductivity. Hence the prediction of temperature at any node of the model requires a 3d thermal FEA Ref [Abaqus, Ansys]. However, the thermal conductivities of the components along the path are approximately constant, so that the temperature rise (above Tc) at any node of the structure is adequately described by a linear sum of contributions from individual sources, i.e:
\begin{equation}
T_i  =   T_\text{c}  +  \sum_{i,j} a_{ij} Q_{j},
\end{equation}

where $Q_j$ is the heat generated at node $j$. In order to extract the matrix of coefficients $a_{ij}$ , the finite element model is run with each heat source (or group of similar sources) switched on in turn with a representative amount of heat, and the temperature rise noted for each of the nodes of interest, which are those in the thermal network model (Figure~\ref{fig:thermalmodel}).  

For a barrel module this gives six sets of node temperatures for each source of injected heat. The thermal impedances in the network are then found from a fit of the linear network to this data. The agreement of the network temperatures using the thermal impedances from the fit with the data from FEA is better than 0.5$^\circ$C for all nodes. This procedure is performed for both, the EOS and the normal module. The thermal impedance from the sensor to the sink ($R_\text{M}+T_\text{C}$) in all cases is between 1.1 and 1.4~$^\circ$C/W, but higher values (between 10 and 20~$^\circ$C/W) are found for other impedances in the network ($R_\text{HCC}$ and $R_\text{FEAST}$), mostly because these are for components with a small footprint constituting a bottleneck for the heat flow.

\textit{[Georg has written the paragraph above to explain the fitting procedure – but is it confusingly different for the petals? I don’t know how he makes the fit].}

There are two recognised departures from linearity of the thermal path: the rise in thermal conductivity of the silicon sensor with decreasing temperature and the rise in heat transfer coefficient (HTC) to the evaporating CO$_2$ coolant with increasing thermal flux. The FEA models are run using mean values for these quantities appropriate to the operating conditions, and the thermoelectric model results are insensitive to the variations expected in practice.

\textit{[GAB: Should we expand on this with plots and tables? There are detailed differences between stave and petal that might be confusing, e.g. re coefficients for the sensor T.      Maybe all we could addin the end is confusion! ].}



