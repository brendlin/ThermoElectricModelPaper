
%% The thermal properties of a silicon detector are typically modeled using numerical methods, such as
%% finite element analysis (FEA) simulation, to determine thermal performance and estimate the risk of
%% thermal runaway. Such methods are essential for understanding detector performance, however they have
%% some limitations: a FEA simulation can only provide results for a discrete set of operating conditions,
%% and the process is computationally expensive.
%
%% A simple analytic model has been developed to complement the FEA approach. This model predicts the
%% behavior of a silicon detector by calculating the cumulative effects of the thermal and electrical
%% characteristics of the on-module detector components. A module's thermal behavior is represented by
%% a simple network of one-dimensional thermal pathways whose properties are taken from FEA simulation.
%% The thermal and electrical properties of front-end electronics are encoded in the model using
%% parametrizations of direct measurements. Using this model, the performance of a detector can be
%% evaluated over a range of operational conditions. The full lifetime of the detector can be simulated
%% by adding the effects of radiation damage and other time-dependent processes.
%
%% We present a working example of the analytic model as applied to the ATLAS ITk strip detector in
%% preparation for the Phase-II Upgrade. The model is used to test design choices, validate
%% specifications, and predict the total power of the strip barrel and endcap subsystems. The model
%% reveals insights into the interplay of detector elements and operational conditions in the silicon
%% module, and it is a valuable tool for estimating the headroom remaining before reaching thermal
%% runaway.

In this paper we discuss the use of linked thermal and electrical network models to predict the behaviour of a complex silicon detector system. We use the silicon strip detector for the ATLAS Phase-II upgrade to demonstrate the application of such a model and its performance. With this example, the thermo-electrical model is used to test design choices, validate specifications, predict key operational parameters such as cooling system requirements, and optimize operational aspects like the temperature profile over the lifetime of the experiment. The model can reveal insights into the interplay of conditions and components in the silicon module, and it is a valuable tool for estimating the headroom to thermal runaway, all with very moderate computational effort.
